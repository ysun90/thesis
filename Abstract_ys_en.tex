\chapter*{Abstract}
\addcontentsline{toc}{chapter}{Abstract}

%\chapter*{Abstracts}
%\label{ch:Abstracts}
%\vspace{-2cm}

%\section*{Abstract}

%\section*{Identification of trapped electron mode in frequency fluctuation spectra of fusion plasma}
%\vspace{-0.3cm}

Turbulence is an important issue in fusion plasmas as it was found to have a direct link to the particles and energy transports, and hence the confinement performance. In this thesis, a systematic study of turbulence has been developed through a parametrization of density fluctuation frequency spectra. The methods has been applied to extract general turbulence properties and its link to particle transport in Ohmic, ion cyclotron resonance heating (ICRH) and lower hybrid (LH) heating plasmas on the Tore Supra tokamak.

The parametrization method is based on a large turbulence database, which includes 350,000 acquisitions from 6,000 discharges and covers the global and local plasma parameters. First, the frequency spectra obtained from core reflectometry measurements are decomposed in four components: the direct current (DC) component, the low-frequency (LF) fluctuations, the broadband (BB) turbulence, and the noise level. Then, for the identified components, different kinds of functions are tested and their fitting performance is analyzed to determine the optimal spectrum parametrization. The DC and LF components were fitted by two Gaussian functions and with a constant parameter for the noise (N) level. In particular, for the BB turbulence, three models are compared qualitatively based on a number of representative spectrum test cases, i.e., the generalized Gaussian, the Voigt, and the Taylor model. In addition, quantitative performance testing is accomplished using the weighted residual sum of squares (RSS) and the Bayesian information criterion (BIC). Both Taylor and generalized Gaussian model are shown to have excellent fitting performance while the Taylor model reflects better physical facts. Both two models have two parameters to determine the curve shape, thus providing high flexibility to spectral shape. In order to reflect simultaneously the features of low-frequency and high-frequency part in frequency spectrum, the fitting has been performed in both linear and logarithmic scale, to which a same weight factor has been given in this study. Furthermore, multiple initial guesses are implemented to make sure a global convergence of the fitting (optimization) process.

The parametrization by the Taylor model is fist applied to Ohmically heated plasmas, and a BB energy basin is systematically observed in the core plasma region, which shrinks with decreasing radial position of the q = 1 surface. This basin might be explained by a drop of the density fluctuation level inside the q = 1 surface. In Ohmic plasmas, the linear Ohmic confinement (LOC) and saturated Ohmic confinement (SOC) regime in Ohmic plasmas are probably dominated by trapped electron modes (TEM) and ion temperature gradient (ITG) modes, respectively. The different confinement regimes might be linked to various turbulence nature which could be reflected by the BB contribution EBB in turbulence spectrum. The same general trend observed recovers in both LOC and SOC regimes. Actually, the data in a fixed radial position are less scatter when distinguishing the two regimes. The BB contribution in SOC is higher than in LOC for each q95 in all radial positions. Here, the basin is defined as the corresponding radial range where the median values are below 0.2. The higher BB turbulence contribution inside the basin in SOC than in LOC for all q95 is remarkable while the difference between the two regimes disappears at highest q95 due the different trend of the BB turbulence contribution with respect to q95 in two regimes. With the increase of q95, the BB turbulence in LOC increases slowly while the BB turbulence decreases rapidly. The difference between LOC and SOC might be related to different turbulence behaviors in the two confinement regimes.

The parametrization is then applied to the additional heating plasmas with ICRH and LH. The evolutions of the broadband contribution have been investigated in ICRH and LH plasmas with increasing heating power under different q95. It is observed that the BB turbulence contribution in ICRH is much higher than in Ohmic plasmas. When the heating power of ICRH is relatively low (0.5 < P < 1.5) under the condition 3 < q95 < 4, the BB basin in the core clearly exist and its location is linked to the q=1 surface. However, the basin becomes weaker rapidly with increasing $P_{ICRH}$ and almost disappears at high Power of ICRH (> 2.5 MW). The phenomenon is valid for all q95, however, the basin is already very shallow at lowest heating power when q95 > 5. The possible existing ICRH power threshold to destabilize the BB turbulence seems to be small (less than 0.5 MW) and depends on q95. For the LH plasmas, the BB basin linked to the q=1 surface remains even at very high power ($P_{LH}$ > 3 MW). The BB turbulence contribution slightly increases in part of the radial positions with increasing Power of LH. At a fixed heating power, EBB inside the q=1 surface slightly increases with increasing q95. The difference behaviors of the BB turbulence in LH and ICRH plasmas are linked to the different turbulence regimes. In low plasma density with LH, trapped electron modes dominates since LH mainly heats electrons with Te>>Ti. In high plasma density with LH, ion temperature gradient begin to play a role with Te~Ti. However, ion temperature gradient always dominates in ICRH plasmas. 