\chapter*{} 
\addcontentsline{toc}{chapter}{Abstracts}

%\chapter*{Abstracts}
%\label{ch:Abstracts}
\vspace{-2cm}

\section*{Abstract}

%\section*{Identification of trapped electron mode in frequency fluctuation spectra of fusion plasma}
%\vspace{-0.3cm}

Turbulence is responsible for anomalous transport, which lowers the particle and energy confinement, and degrades the performance of fusion devices. Several types of instabilities can induce turbulent phenomena. In toroidal magnetized plasmas, turbulence is triggered by micro-instabilities. At low wave-numbers $k_{\perp}$ normalized to the Larmor radius $\rho_i$ ($0.1<k_{\perp}\rho_i<1$), the dominant micro-instabilities are the Ion Temperature Gradient (ITG) modes and the Trapped Electron Modes (TEM). 

An important property of these electrostatic modes is that they both exhibit a threshold above which they are unstable. As ITG and TEM are driven by different temperature and density gradients, they can either (i) both be stable, (ii) both coexist or (iii) give way to a single dominant mode (ITG or TEM). 

The heat and particle transport induced by ITG and TEM are noticeably different. First the sign of their convection velocity can be opposite, leading to an outward/inward particle pinch, which tends to flatten/peak the density profiles, respectively. ITG and TEM can also influence the intrinsic toroidal velocity in a different way via the residual stress mechanisms. Additionally, ITG and TEM show distinct dependencies on plasma parameters (magnetic shear, $T_i/T_e$, collisionality, etc.). Therefore changes in these parameters can have different effects whether ITG or TEM dominates. For these reasons it is important to be able to discriminate experimentally the regimes dominated by TEM and ITG.

These modes can be distinguished via their phase velocity at a given wave-number in the plasma frame, which is in opposite directions. However this is challenging since the $E\times B$ drift of the plasma frame is expected at much higher values than the phase velocity which is not directly measurable. Alternatively, transport studies can help discriminate ITG from TEM dominated regimes, but it requires perturbation experiments to get a precise estimation of the diffusion coefficients and the convection velocities.

This thesis shows that the analysis of frequency fluctuation spectra can provide an additional experimental indication of the dominant mode. Depending on the plasma scenario, fluctuation spectra can display different frequency components:
\begin{itemize}
	\item Broadband spectra ($\Delta f\approx$ hundreds of kHz) which are always observed. Their amplitude is maximum at the zero frequency and they are attributed to turbulence.
	\item Coherent modes ($\Delta f\approx1$ kHz) which oscillate at a very well defined frequency. They can for example be due to geodesic acoustic or magnetohydrodynamic (MHD) modes
	\item Quasi-Coherent (QC) modes ($\Delta f\approx$ tens of kHz) which oscillate at a rather well defined frequency but which are reminiscent of broadband fluctuations.
\end{itemize}
The fluctuation study performed in the plasma core region shows that the fluctuation spectra in TEM-dominated regimes can be noticeably different from the ones in ITG-dominated regimes, as only TEM can induce QC modes. Such a finding has been achieved by comparing fluctuations measurements with simulations.

Measurements are made with a reflectometry diagnostic, a radar-like technique able to provide local indications of the density fluctuations occurring in the vicinity of the reflection layer. Frequency fluctuation spectra are inferred from a Fourier analysis of the reflectometry signal.

First, the main properties of QC modes are characterized experimentally. Their normalized scale is estimated to $k_{\perp}\rho_i\leq1$, their amplitude is ballooned on the low field side mid-plane and they can be observed at many different radii. These indications are in agreement with what could be expected for ITG/TEM instabilities.

Then reflectometry measurements are analyzed in Ohmic plasmas. QC modes are observed in the Linear Ohmic Confinement (LOC) regime dominated by TEM whereas only broadband spectra are seen in the Saturated Ohmic Confinement (SOC) regime dominated by ITG. Frequency spectra from nonlinear gyrokinetic simulations show that TEM induce a narrow frequency spectra responsible for the QC modes observed experimentally. This interpretation of the measured spectra is made via a synthetic reflectometer diagnostic using the gyrokinetic simulations as an input. The QC modes observed in the plasma core have then been renamed QC-TEM as a reference of their TEM origins.

Thereafter, the first applications of the knowledge of QC-TEM properties are made in Ohmic plasmas of Tore Supra, TEXTOR, JET and ASDEX-Upgrade. The global disappearance of QC-TEM simultaneously with the LOC-SOC transition suggests that the stabilization of TEM plays an important role in the change of Ohmic regime. The disappearance of QC-TEM can also be correlated to intrinsic toroidal velocity bifurcation, which is not explained by neoclassical predictions.

Another application using the QC signature of TEM has been done in Tore Supra ECRH plasmas. A previous study found an increase of the diffusion coefficient with the electron temperature gradient in a region predicted to be dominated by electron modes ($r/a<0.2$). Further out ($r/a>0.2$), the diffusion was independent of the electron temperature gradient in a region dominated by ion turbulence. Reflectometry measurements brought an additional indication by showing the presence of QC-TEM at $r/a<0.2$ and a broadband spectrum at $r/a>0.2$, supporting the previous investigations. 

Finally, transitions between electrostatic fluctuations (QC-TEM) and electromagnetic MHD modes have been observed.

Spatial transitions from TEM toward MHD modes are reported in Ohmic ASDEX-Upgrade plasmas and Tore Supra plasmas heated with Lower Hybrid (LH) waves. They may contribute to the sudden stabilization of TEM observed toward the plasma center.

Temporal interplay between QC-TEM and MHD modes has also been observed with various heating schemes in Tore Supra plasmas (LH and electron cyclotron resonance heating), and in JET with neutral beam injection heating. This interplay which may have different drives (sawtooth, magnetic shear) indicates that QC-TEM and MHD are anti-correlated, QC-TEM fluctuations showing a delay of the order of ms compared to the MHD modes.

These multiple observations suggest that spatial and temporal interactions between MHD and turbulent instabilities may be at play in the plasma core region.




%\section*{R\'esum\'e}
%\vspace{-0.2cm}

\section*{R\'esum\'e}
La turbulence est responsable du transport anormal qui diminue le confinement du plasma et par conséquent dégrade les performances des réacteurs à fusion. Différent types d’instabilités peuvent induire de la turbulence. Dans les plasmas magnétisés toroïdaux, la turbulence est déclenchée par des micro-instabilités. A faibles nombres d’ondes $k_{\perp}$ normalisés au rayon de Larmor $\rho_i$ ($0.1<k_{\perp}\rho_i<1$), les micro-instabilités dominantes sont les "Ion Temperature Gradient" (ITG) et les "Trapped Electron Modes" (TEM).

Ces modes ont un seuil au-delà duquel ils sont instables. Comme les ITG et les TEM sont déstabilisés par différents gradients, ils peuvent tous les deux être (i) stables, (ii) instables, ou (iii) donner un seul mode dominant (ITG ou TEM).

Le transport de chaleur et de particules induit par les ITG et les TEM est sensiblement différent. Tout d’abord, le signe de leur vitesse de convection peut-être opposé, contribuant à piquer ou aplatir les profils de densité. Les ITG et les TEM peuvent aussi influencer la vitesse de rotation toroïdale intrinsèque de manière différente. De plus, les ITG et les TEM dépendent différemment de plusieurs paramètres du plasma, des modifications de ces paramètres peuvent avoir des effets différents si les ITG ou les TEM dominent. Pour ces raisons, il est important d’être capable de discriminer expérimentalement les régimes dominés par les TEM ou les ITG.

Ces modes peuvent être distingués par leur vitesse de phase à un nombre d’onde donné dans le référentiel du plasma, car elles sont opposées. Cependant, cela est difficile car la dérive $E\times B$ du référentiel du plasma est nettement supérieure à la vitesse de phase qui ne peut pas être mesurée directement. Alternativement, l’étude du transport peut permettre de discriminer les régimes dominés par les TEM de ceux dominés par les ITG, mais cela nécessite des expériences de perturbation pour avoir une estimation précise des coefficients de diffusion et de convection.

Cette thèse montre qu’une analyse des spectres fréquentiels de fluctuation peut fournir une indication expérimentale du mode dominant. En fonction du scénario de plasma, les spectres fréquentiels de fluctuation peuvent exhiber différentes composantes :
\begin{itemize}
	\item Les spectres à large-bande ($\Delta f\approx$ quelques centaines de kHz) ont une amplitude maximum à la fréquence zéro et sont atribués à la turbulence
	\item Les modes cohérents ($\Delta f\approx1$ kHz) qui oscillent à une fréquence bien définie peuvent par exemple être dus aux modes magnétohydrodynamiques (MHD).
	\item Les modes Quasi-Cohérents (QC) ($\Delta f\approx$ quelques dizaines de kHz) qui oscillent à une fréquence plutôt bien définie mais qui rappellent les spectres à large-bande de par leur large bande spectrale.
\end{itemize}
Cette étude des fluctuations menée dans le plasma de cœur montre que contrairement aux spectres des ITG, les spectres des TEM peuvent induire des modes QC. Cette découverte a été faite en comparant les fluctuations mesurées avec celles fournies par des simulations.

Les mesures ont été faites par réflectométrie, un diagnostic de type radar capable de fournir des indications locales sur les fluctuations de densité dans la région de la couche de coupure. Les spectres fréquentiels de fluctuation sont obtenus par une analyse de Fourier des signaux de réflectométrie. Les principales propriétés des modes QC ont tout d’abord été caractérisées expérimentalement. Leur échelle normalisée est estimée à $k_{\perp}\rho_i \leq 1$, leur amplitude est ballonnée côté faible champ et ils peuvent être observés à de nombreux rayons différents. Ces indications sont en accord avec ce que l’on peut attendre pour les instabilités de type ITG/TEM.

Puis, les mesures de réflectométrie ont été analysées dans les régimes de Confinement Ohmique Linéaire (LOC) dominé par les TEM et de Confinement Ohmique Saturé (SOC) dominé par les ITG. Les modes QC ont été observés seulement en régime LOC alors qu'un spectre à bande large a été observé en régime SOC. Les simulations gyrocinétiques non-linéaires montrent que les TEM induisent des spectres fréquentiels étroits responsables des modes QC observés expérimentalement. Cette interprétation a été faite via un réflectomètre synthétique utilisant les simulations gyrocinétiques. Les modes QC observés dans le cœur du plasma ont été renommés QC-TEM en référence à leur lien avec les TEM.

Les premières applications de la connaissance des QC-TEM ont été faites dans les régimes Ohmiques de Tore Supra, TEXTOR, JET et ASDEX-Upgrade. Ils indiquent que la disparition des QC-TEM est globale et simultanée à la transition LOC-SOC, suggérant que la stabilisation des TEM joue un rôle important dans le changement de régime Ohmique. La disparition des QC-TEM peut être aussi corrélée à des bifurcations de la vitesse toroïdale intrinsèque qui ne sont pas expliquées par des prédictions néoclassiques.

Un autre application de la connaissance des QC-TEM a été faite dans les plasmas de Tore Supra chauffés par ECRH. Une étude précédente avait montré que dans une région dominée par une turbulence électronique ($r/a<0.2$), une augmentation du coefficient de diffusion était observée lorsque le gradient de température électronique normalisé augmentait. Plus vers l’extérieur ($r/a>0.2$), une diffusion indépendante du gradient de température électronique normalisé avait été observée alors qu’une turbulence ionique dominait. Les mesures de réflectométrie ont amené une indication supplémentaire en montrant des QC-TEM seulement pour $r/a<0.2$ et un spectre à bande large pour $r/a>0.2$, appuyant les précédentes études.

Finalement, des transitions entre des fluctuations électrostatiques (QC-TEM) et électromagnétiques (MHD) sont rapportées.
Radialement, les TEM peuvent disparaitrent au profit de modes MHD dans des plasmas Ohmiques d’ASDEX-Upgrade et dans les plasmas de Tore Supra utilisant le chauffage hybride. Temporellement, des interactions entre des modes MHD et des TEM ont aussi été rapportées dans divers régimes de chauffages électroniques de Tore Supra et dans des plasmas de JET chauffés par injection de neutres. Ces observations suggèrent que des interactions spatiales et temporelles peuvent avoir lieu entre des instabilités turbulentes et des modes MHD dans le cœur du plasma.




\section*{Overzicht}

Turbulentie is verantwoordelijk voor het anomalous transport dat de deeltjes- en energieconfinement vermindert en de performantie van fusie-reactoren verslechtert. Verscheidende types van onstabiliteiten kunnen turbulente fenomenen induceren. In toroïdale magnetische geometrieëen wordt turbulentie getriggerd door micro-onstabiliteiten. Voor lage waardes voor de wave numbers $k_{\perp}$, genormalizeerd tot de Larmor-straal $\rho_i$ ($0.1<k_{\perp}\rho_i<1$), is de dominante micro-onstabiliteit de Ion Temperature Gradient Mode (ITG) en de Trapped Electron Mode (TEM). Een belangrijke eigenschap van deze electrostatische modes is dat beide een grenswaarde hebben waarboven ze onstabiel worden. Daar ITG en TEM gedreven worden door verschillende gradiënten kunnen ze ofwel (i) allebei stabiel zijn, (ii) beide bestaan of (iii) resulteren in één enkele dominante mode.

Het warmte- en deeltjestransport geïnduceerd door ITG en TEM is erg verschillend. Ten eerste kan het teken van de convectiesnelheid omgekeerd zijn, wat tot een outward/inward particle pinch kan leiden wat op zijn beurt een vlak of gepiekt dichtheidsprofiel kan veroorzaken. ITG en TEM kunnen ook de intrensieke toroïdale snelheid op een verschillende manier beïnvloeden via residuele spanningsmechanismen. Bovendien kunnen ITG en TEM verschillende afhankelijkheden vertonen van de plasmaparameters (magnetische shear, $T_i/T_e$, collisionality, etc.). Daarvoor kunnen veranderingen in deze parameters verschillende effecten hebben naargelang ITG of TEM domineert. Omwille van al deze redenen is het belangrijk om experimenteel te kunnen bepalen welke regimes gedomineerd worden door TEM en ITG.

De modes kunnen worden ondersheiden met behulp van hun fasesnelheden voor een gegeven golfgetal in het plasma frame, die in verschillende richtingen georiënteerd zijn. Maar dit is een uitdaging omdat de $E\times B$ drift van het plasma frame verwacht wordt bij veel grotere waardes van de fasesnelheid die niet direct meetbaar is. Als een alternatief kunnen transportstudies helpen om te discrimeneren tussen ITG- en TEM-gedomineerde regimes, maar hiervoor zijn perturbatieëxperimenten noodzakelijk om een goede schatting te kunnen geven van de diffusiecoefficiënten en de convectiesnelheden.

Deze thesis illustreert dat de analyse van de spectra van de frequentiefluctuaties een verdere experimentele indicatie kan geven van de dominante mode. Afhankelijk van het scenario kunnen deze spectra verschillende componenten vertonen:

\begin{itemize}
	\item Breedbandige spectra ($\Delta f\approx$ honderden kHz) worden altijd geobserveerd. Hun amplitude is maximimaal voor de nulfrequentie en ze worden veroorzaakt door turbulentie.
	\item Coherente modes ($\Delta f\approx1$ kHz) oscilleren met een scherp gedefinieerde frequentie. Ze kunnen bijvoorbeeld het gevolg zijn van een geodesische acoustische mode of mangetohydrodynamische modes.
	\item Quasi-Coherente (QC) modes ($\Delta f\approx$ tientallen kHz) oscilleren met een nauwkeurig gedefinieerde frequentie maar ze zijn reminiscent of the broadband fluctuations (?).
\end{itemize}

Het toepassen van deze fluctuatiestudie in de plasma core leidt tot de vaststelling dat de spectra van TEM modes zeer verschillend kunnen zijn van een ITG mode, omdat enkel TEM QC modes kan induceren. Deze ontdekking wordt gedaan door het vergelijken van fluctuatiemetingen met simulaties. Metingen worden gedaan met behulp van reflectometrie, een techniek die op radar lijkt en locale indicaties geeft over de dichtheidsfluctuaties dichtbij de reflectielaag. Frequentiefluctuatiespectra worden dan afgeleid door een analyse van Fourier van het reflectometriesignaal.

De voornaamste kenmerken van QC modes worden eerst experimenteel bestudeerd. Hun genormalizeerde schaal wordt geschat op $k_{\perp}\rho_i\leq1$, hun amplitude balloont in de low-field side midplane en ze kunnen worden geobserveerd op verscheidende radii. Deze aanwijzingen zijn coherent met wat verwacht kan worden voor ITG/TEM instabiliteiten.

Daarna worden reflectometriemetingen geanalyzeerd in Ohmische plasmas. QC modes worden geobserveerd bij Linear Ohmic Confinement (LOC), gedomineerd door TEM, maar enkel een breedband spectrum wordt gezien in bij Saturated Ohmic Confiment (SOC), gedomineerd door ITG. Frequentiespectra van niet-lineaire gyrokinetische simulaties tonen aan dat TEM het smal frequentiespectrum kan induceren dat verantwoordelijk kan zijn voor de QC modes die experimenteel worden geobserveerd. Deze interpretatie van de gemeten spectra gebeurt via een synethetische reflectometer dat de gyrokinetische simulaties als input heeft. De QC modes die geobserveerd worden in de plasma core worden dan QC-TEM modes genoemd, met de referentie naar hun TEM-oorsprong. Dan worden de eerste toepassingen van de kennis over QC-TEM gedaan in de Ohmische plasmas van de Tore Supra, TEXTOR, JET en ASDEX-Upgrade. De globale verdwijning van QC-TEM gelijktijdig met de LOC-SOC transitie suggereert dat de stabilizatie van de TEM een grote rol speelt in de verandering van het Ohmische regime. De verdwijning van de QC-TEM kan ook gecorreleerd zijn met de intrinsieke toroïdale snelheidsbifurcatie die niet worden verklaard met neoklassieke voorspellingen.

Een andere toepassing die gebruik maakt van de QC-natuur van TEM wordt gedaan in Tore Supra ECRH plasmas. Een vorige studie toonde aan dat de diffusiecoefficiënt stijgt als functie van de electrontemperatuursgradiënt in een regio die wordt voorspeld gedomineerd te zijn door electron modes ($r/a<0.2$). Verder naar buiten ($r/a>0.2$) was de diffusie onafhankelijk van de  electrontemperatuursgradiënt in een regio die gedomineerd werd door ion turbulence. Reflectometrie metingen geven een bijkomende indicatie door aanwezigheid aan te tonen van QC-TEM in $r/a<0.2$ en een breedbandspectrum voor $r/a>0.2$, wat de resultaten van eerdere onderzoeken steunt.

Uiteindelijk werden transities tussen electrostatische turbulentie (QC-TEM) en electromagnetische MHD modes geobserveerd.

Ruimtelijke transities van TEM naar MHD modes werden gereporteerd voor Ohmische ADEX-Upgrade plasmas en Tore Supra plasmas die werden verwarmd door lower hybrid (LH) waves. Deze zouden kunnen bijdragen tot de plotse stabilizatie van TEM die wordt geobserveerd dicht bij het centrum van het plasma. Tijdelijke interplays tussen QC-TEM en MHD modes werden ook geobserveerd met meerdere verwarmingsschemas in Tore Supra plasmas (LH en electron cyclotron resonance heating) en JET (NBI). Deze interplays, die verschillende drives kunnen hebben (sawtooth, magnetic shear) tonen aan dat QC-TEM en MHD anti-gecorreleerd zijn, waarbij QC-TEM fluctuaties een vertraging van de orde van ms vertoont vergeleken met MHD modes.

Dit veelvoud van observaties suggereert dat ruimtelijke en tijdelijke interacties tussen MHD en turbulente onstabiliteiten van belang zijn in de plasma core region.
