\chapter*{Acknowledgements}
\addcontentsline{toc}{chapter}{Acknowledgements}


During the last three years, many colleagues and other PhD students have given me lots of help and suggestions to my research work. I have gained valuable knowledge and experience from them, which are much more than the completion of this thesis.

First, I would like to give my thanks to my supervisors. Most of the time I work in IRFM, CEA Cadarache with Dr. Roland Sabot, as my thesis director, who has guided my work from the very beginning when he explained the details of the reflectometry and other diagnostics on Tore Supra to me with great patience. Even when I am in Belgium for months, we still has a regular meeting to discuss my progress. In particular, I am most impressed by his strong passion towards the scientific research and the interests to explore new problems. When I stayed in Ghent University in my first year to follow some courses, I worked directly with Dr. Gr\'{e}giore Hornung and also under the supervision of Prof. dr. Geert Verdoolaege. Their initial envisions of applying the data science in the study of plasma turbulence has driven the systematic turbulence study of this thesis. Although Gr\'{e}giore has left the fusion area after one year, we still stay in touch and he has been always interested in my work. Geert has been giving me suggestions in many aspects including globally the research direction to the very detailed data analysis techniques through my thesis. Even though I was not in Belgium any more for the last two yeas, we has been staying in close touch by regularly meeting together to discuss my work. In particular, Geert has helped to modify and improve my every manuscript with extreme preciseness and patience. As my thesis promoter, he also has many more promotions of the research direction to be carried out, but unfortunately many ideas have not be realised due to the limited time given for the PhD. As another director of my thesis, Prof. St\'{e}phane Heuraux has been giving instructions in a global viewpoint. Although my stay in University of Lorraine is relatively short, I really appreciated the delight working time with him in Nancy. When most of the time we work together remotely, he always replied to my any questions by the quickest time with the most clear and detailed explains when he is very busy. In addition, since usually I was not in the University, he has helped me to handle with some complicated documents especially at the beginning when the registration documents are all written in French which I knew very little then. I also have another mentor in IRFM-CEA, Dr. S\'{e}bastien Hacquin, who has worked temporarily in Culham since my second year. Although being very busy and has one-hour time difference with me, he still deeply cares about my research work. He would like to know my progress regularly and read each of my manuscript very carefully with detailed comments.


Then, I wish to thank the colleagues who are willing to become the jury members my defense: Dr. Pascale Hennequin from \'{E}cole Polytechnique, Prof. Stefano Coda from EPFL, Prof. Michel Vergnat from University of Lorraine and Prof. Filip De Turck and Prof. Toon Verstraelen both from Ghent University. Thank you very much for spending the time to read my thesis and give the suggestions! Dr. Pascale Hennequin has also given me helpful instructions on my work on the development of the parametrization method.


In IRFM-CEA, where I have spent most of my PhD time, I hope to thank all the colleagues who have participated in the operation of Tore Supra and especially the different diagnostics. Without your previous excellent work, my systematic (database) study could not be realised. The following colleagues have some specific contributions to my PhD work. I discussed with L. Vermare about spectrum fitting by the Taylor model. The analysis tool provided by D. Elbeze was helpful in validating the position of the sawteeth inversion. The discussions with P. Devynck, J. F. Artaud and C. Bourdelle help to further understand the effects of $Z_\mathrm{eff}$ on turbulence as well as the more accurate estimation of $Z_\mathrm{eff}$ by the scaling law. X. Garbet motivates the latter stage of the research work based on the collisional effects on the micro-instabilities. Many colleagues gave me a lot of help in many aspects including: C. Amador, X. Zou, F. Clairet, C. Bottereau, J. C. Giaccalone, R. Guirlet, C. Gil, Ph. Moreau and L. Lu. I would like to express special thanks to Ph. Lotte and F. Imbeaux who have given me numerous help for my stay in IRFM, especially approving me to many important conferences during my PhD time. The two secretaries in our group, N. Borio and V. Icard also helped me to handle with many document work.


When studying in Ghent University, many colleagues and students gave me different helps: Kathleen Van Oost, Frank Janssens, Eveline Indemans, Prof. Jean-Marie Noterdaeme and Prof. Christophe Leys. I also wold like to thank many other PhD students with whom I have shared many unforgettable moments during my PhD: Rennan Morales, Iaroslav Morgal, Tianbo Wang, Lei Wang, Sundaresan Sridhar, Song Xiao, Rui Mao, Zhaoxi Chen, Georgiy Zadvitskiy, Arvydas and Nicolas Ialovega.


Thanks for the funding provided by the International Doctoral College in Fusion Science and Engineering (FUSION-DC), CEA Cadarache and Ghent University!


Finally, I express my great thank to my families including my father, mother, mother-in-law, my father-in-law and my wife. Without your support I would not be able to complete this thesis. Especially, my wife has also helped to improve this manuscript.


Thank you for everyone again!
















