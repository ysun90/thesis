
\chapter*{}

\cftaddtitleline{toc}{chapter}{Abstract/R\'{e}sum\'{e}}{}
%\addcontentsline{toc}{chapter}{Abstract/R\'{e}sum\'{e}}


\section*{Abstract}


To systematically study the plasma turbulence in tokamaks, a parametrization method of frequency spectra from the Tore Supra reflectometry database has been developed. The database includes $350,000$ acquisitions obtained from $6,000$ discharges with different heating scenarios. In the parametrization, each spectrum has been decomposed into four component: the direct current component, the low-frequency (LF) fluctuations, the broadband (BB) turbulence and the noise level, and each component has been fitted by a function. Specifically for the BB turbulence component, three different functions have been tested and compared. The generalized Gaussian function and the Taylor function have shown excellent fitting performance. The radial profiles of the BB contribution ($E_\mathrm{BB}$) with different edge safety factor have been investigated. In Ohmic plasmas, $E_\mathrm{BB}$ in the saturated Ohmic confinement regime is observed to be systematically higher than in the linear Ohmic confinement regime. In L-mode plasmas, $E_\mathrm{BB}$ in the ion cyclotron resonance heating plasmas is globally much higher than in the lower hybrid heating plasmas. To understand the observations, the collisional effects on the modifications of the spectra have also been studied. With the previous kinetic simulation results, a change of the dominating micro-instabilities, i.e., between the trapped electron modes and the ion temperature gradients modes, has been proposed to explain the behaviors of $E_\mathrm{BB}$, supported by further database analysis of the density peaking and the LF component. This database study of plasma turbulence motivates more detailed studies by full-wave and gyrokinetic simulations, in order to confirm the link between the modifications of spectra and the micro-instabilities for both Ohmic and L-mode plasmas.


\section*{R\'{e}sum\'{e}}



L'\'{e}tude syst\'{e}matique de la turbulence plasma dans les tokamaks, via une m\'{e}thode de param\'{e}trisation des spectres de fr\'{e}quence extraits de la base de donn\'{e}es de r\'{e}flectom\'{e}trie de Tore Supra a \'{e}t\'{e} effectu\'{e}e. Cette base de donn\'{e}es est constitu\'{e}e de 350 000 acquisitions obtenues \`{a} partir de 6 000 chocs incluant diff\'{e}rents sc\'{e}narios de chauffage. La param\'{e}trisation consiste en une d\'{e}composition de chaque spectre en quatre composantes: la composante dite continue, les fluctuations de basse fr\'{e}quence (BF), la turbulence \`{a} large bande en fr\'{e}quence (BB) et le niveau de bruit o\`{u} chaque composante a \'{e}t\'{e} approch\'{e}e par une fonction pr\'{e}d\'{e}fine. Pour la composante de turbulence BB, trois fonctions diff\'{e}rentes ont \'{e}t\'{e} test\'{e}es et compar\'{e}es. La fonction gaussienne g\'{e}n\'{e}ralis\'{e}e et la fonction de Taylor ont montr\'{e} d'excellentes performances dans la plupart des cas. Les profils radiaux de la contribution BB ($E_\mathrm{BB}$) pour diff\'{e}rents facteurs de s\'{e}curit\'{e} ont \'{e}t\'{e} \'{e}tudi\'{e}s. Dans les plasmas ohmiques, $E_\mathrm{BB}$ dans le r\'{e}gime de confinement ohmique satur\'{e} est syst\'{e}matiquement plus \'{e}lev\'{e} que dans le r\'{e}gime de confinement ohmique lin\'{e}aire. Dans les plasmas en mode bas confinement L, $E_\mathrm{BB}$ dans les plasmas chauff\'{e}s par une onde  \`{a} la r\'{e}sonance cyclotron ionique est globalement beaucoup plus \'{e}lev\'{e} que dans les plasmas chauff\'{e}s par onde hybride inf\'{e}rieur. Pour comprendre les observations, les effets des collisions sur les modifications du spectre ont \'{e}galement \'{e}t\'{e} introduits. En utilisant les r\'{e}sultats de simulations cin\'{e}tiques, il a \'{e}t\'{e} propos\'{e} que les micro-instabilit\'{e}s dominantes, c'est-\`{a}-dire entre les modes d'\'{e}lectrons pi\'{e}g\'{e}s et les modes de gradients de temp\'{e}rature des ions, suffisent pour expliquer le comportement de $E_\mathrm{BB}$, ceci \'{e}tant confort\'{e}s par une analyse de la base de donn\'{e}es de la forme du profil de densit\'{e} et de la composante LF. Cette base de donn\'{e}es sur la turbulence plasma incite \`{a} poursuivre des \'{e}tudes plus d\'{e}taill\'{e}es par simulations des r\'{e}sultats de r\'{e}flectom\'{e}trie et gyrocin\'{e}tiques, afin de confirmer le lien entre les modifications du spectre et les micro-instabilit\'{e}s des plasmas \`{a} la fois en mode ohmique et en mode L.


\thispagestyle{empty}