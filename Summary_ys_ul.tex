\chapter*{Summary}
\addcontentsline{toc}{chapter}{Summary}

{\Large

Magnetic confinement nuclear fusion research aims at the development of a clean, safe and inexhaustible source of baseload electric power. Based on the energy released during fusion of light nuclei of hydrogen isotopes in a hot tokamak plasma, its realization depends crucially on the ability to regulate the transport of energy and particles through the plasma. It is now known that the main contribution to transport in tokamak plasmas is due to small-scale turbulence, hence understanding and control of the turbulent transport is key to achieve the desired confinement properties. In the core region of tokamak plasmas, drift-wave turbulence is mainly caused by two types of micro-instabilities: the so-called trapped electron modes (TEM) and the ion temperature gradient (ITG) modes. Considerable research efforts are directed towards understanding turbulence properties under varying plasma conditions. The degree of collisionality of the plasma is known to affect the dominating instability, since particles trapped in their motion by the magnetic well tend to be detrapped by collisions. Shedding more light on these mechanisms is one of the key purposes of this thesis.

Among the different turbulence diagnostic tools, reflectometry is a radar-like technique, used in this PhD work, which is based on detecting the properties of microwaves reflected by the plasma. Reflectometry allows to detect fluctuations of the plasma density with a high spatial resolution. In particular, standard fixed-frequency reflectometry has been extensively utilized to extract spectral characteristics and correlation properties of turbulent fluctuations, as well as their link with the micro-instabilities and the turbulent transport.

In fusion science, most experimental studies are set up according to a carefully chosen set of plasma conditions, and attempt to vary, in a controlled way, one or a few plasma parameters at a time, in order to investigate the effect on some plasma phenomenon of interest. In contrast, in this work a large database of reflectometry measurements from the Tore Supra tokamak has been created, and specialized tools have been developed to detect patterns in the data, persisting across a broad range of plasma conditions. The database includes 350,000 spectra from 6,000 discharges in basic Ohmic operation, and in the so-called low-confinement mode (L-mode) plasmas, which are heated by means of auxiliary heating systems. The aim was to extract general trends of turbulence properties, and link them to the occurrence and behavior of the micro-instabilities driving the turbulence, hence contributing to the understanding of plasma turbulence.

To accomplishing these goals, a key ingredient is a proper quantification of the fixed-frequency reflectometry power spectra. A robust spectrum quantification scheme enables standardization, which in turn allows the kind of systematization envisaged in this work. Each spectrum in our Tore Supra database was decomposed into four components: the direct current (DC) component, the low-frequency (LF) fluctuations, the broadband (BB) turbulence and the noise level. Various parametrization functions were tested and compared to determine the optimal spectrum fit, with adequate robustness properties in the presence of a large variety of spectrum shapes and plasma conditions. The components of the spectra at low frequencies, including the DC and LF components, were fitted by two Gaussian functions, and the noise was fitted by one constant parameter. The BB component represents the energy distribution of turbulence in the frequency domain and may assume a variety of shapes (Gaussian, Lorentzian, etc.) under different plasma conditions. For the BB component, three different fit models (generalized Gaussian, Voigt and Taylor) were compared quantitatively using a large number of spectra. This indicated excellent performance of the generalized Gaussian (GG) model, followed by the Taylor model. On the other hand, the parameters of the Taylor model are more amenable to physical interpretation.

Equipped with a robust spectrum parametrization method, the spectral parameters (width, shape and contribution) of the BB and LF components from both the GG and the Taylor model provide quantitative information about the turbulence properties. The most straightforward spectral characteristic is the BB contribution of the spectra ($E_\mathrm{BB}$). The full radial profile of the BB component has been investigated at different edge safety factors ($q_{\psi}$), a crucial dimensionless parameter determining the plasma stability, in both Ohmic and L-mode plasmas. In Ohmic plasmas, a remarkable reduction of $E_\mathrm{BB}$, referred to as the $E_\mathrm{BB}$ basin, was systematically observed near the central region. Further investigation revealed a direct link between the $E_\mathrm{BB}$ basin and the $q = 1$ surface. This is related to the occurrence of magnetohydrodynamic instabilities known as sawteeth. Specifically, $E_\mathrm{BB}$ was considerably reduced ($E_\mathrm{BB} < 0.2$) inside the $q = 1$ surface and the width of the $E_\mathrm{BB}$ basin was found to be approximately proportional to the $q = 1$ position. Outside the $q = 1$ surface, $E_\mathrm{BB}$ increases above $0.5$ at both the low-field-side and high-field-side, however with a strong asymmetry. Discriminating between the linear Ohmic confinement (LOC) regime and the saturated Ohmic confinement (SOC) regime, it was discovered that $E_\mathrm{BB}$ in the SOC regime is systematically higher than in the LOC regime, throughout the entire plasma cross-section. In L-mode, we focused on plasmas with pure lower hybrid (LH) heating or pure ion cyclotron resonance heating (ICRH). In pure LH-heated plasmas, the $E_\mathrm{BB}$ basin was observed as well, at different heating powers ($P_\mathrm{heat}$). However, with the same $P_\mathrm{heat}$ in pure ICRH plasmas, $E_\mathrm{BB}$ was found to be greatly enhanced, with a weak or disappearing broadband basin.

In the final part of the work, interpretation of the observations of the spectral trends across the database was treated. We focused on the collisional effects on different spectral characteristics in the various confinement regimes. In Ohmic plasmas, a general increase of $E_\mathrm{BB}$ with collisionality was observed at all radial positions. This global trend was seen to be consistent with gyrokinetic simulations reported in the literature. Specifically, a wider BB component was observed in the SOC regime (higher density or collisionality), compared to the LOC regime. This correspondence suggests a possible interpretation of the trends of the broadband width in terms of a transition of the dominating instability driving the turbulence. In particular, the TEM and ITG instabilities have been linked with the LOC and the SOC regimes, respectively. This possible interpretation was supported by further analysis of the LF component and the density peaking. In addition, other BB characteristics (width and shape) have also been studied to obtain a deeper understanding. In L-mode plasmas, a similar trend of $E_\mathrm{BB}$ with collisionality was observed, suggesting a similar interpretation as in the Ohmic case. Again, this was supported by a global analysis of the LF component and the density peaking.

The main contribution of the present work has been to extend analysis of turbulence characteristics from reflectometry spectra to a broad range of plasma conditions in a large database. The standardization offered by this approach has enabled a systematic study of turbulence properties across the database. Various patterns were observed and could be linked to experimental observations or simulations carried out using a much reduced set of plasma conditions. This is the first demonstration in fusion science of systematic characterization of turbulence properties in such a large database. Additional confirmation of the link between the spectral trends and the dominating instability is to be provided by full-wave and gyrokinetic simulations. However, the present study has allowed to establish a number of important, robust trends of turbulence properties in tokamak plasmas, thus putting their interpretation on firmer ground.

}

\newpage


\section*{{\Large R\'{e}sum\'{e} \'{e}tendu}}

{\Large

La recherche sur la fusion nucl\'{e}aire par confinement magn\'{e}tique a pour but de d\'{e}velopper une source d'\'{e}nergie \'{e}lectrique propre, s\^{u}re et in\'{e}puisable. Bas\'{e} sur l'\'{e}nergie r\'{e}sultant de la fusion de noyaux d'isotopes d'hydrog\`{e}ne se produisant dans un plasma magn\'{e}tis\'{e} chaud, la viabilit\'{e} de cette source d'\'{e}nergie d\'{e}pend, de mani\`{e}re cruciale, de la ma\^{\i}trise du transport d'\'{e}nergie et de particules \`{a} travers le plasma. Il est maintenant connu que la principale contribution au transport dans les plasmas de tokamak est li\'{e}e \`{a} la turbulence \`{a} petite \'{e}chelle. Il est donc essentiel de comprendre et de contr\^{o}ler le transport turbulent pour obtenir les propri\'{e}t\'{e}s de confinement souhait\'{e}es. Dans la r\'{e}gion centrale des plasmas de tokamak, la turbulence associ\'{e}es aux ondes de d\'{e}rive est principalement reli\'{e}e \`{a} deux types de micro-instabilit\'{e}s: les modes appel\'{e}s: mode d'\'{e}lectrons pi\'{e}g\'{e}s (TEM) et les modes \`{a} gradient de temp\'{e}rature ionique (ITG). Beaucoup efforts en recherche sur les plasmas de fusion sont d\'{e}volus \`{a} la compr\'{e}hension des propri\'{e}t\'{e}s de la turbulence pour diverses conditions plasma. Les collisions impactent l'instabilit\'{e} dominante et le transport parce que d'une part les particules confin\'{e}es dans la cage magn\'{e}tique ont tendance \`{a} \^{e}tre \'{e}ject\'{e}es par les collisions et d'autre part les collisions r\'{e}duisent le nombre effectif de particules \`{a} l'origine des instabilit\'{e}s. La mise en \'{e}vidence des diff\'{e}rents r\'{e}gimes de turbulence li\'{e}s aux diff\'{e}rentes instabilit\'{e}s, et donc de transport, est l'un des objectifs principaux de cette th\`{e}se.

Parmi les diff\'{e}rents moyens de diagnostic de la turbulence, la r\'{e}flectom\'{e}trie est une technique, de type radar, utilis\'{e}e dans cette th\`{e}se. Celle-ci repose sur la d\'{e}tection des propri\'{e}t\'{e}s des micro-ondes r\'{e}fl\'{e}chies par le plasma. La r\'{e}flectom\'{e}trie permet d'acc\'{e}der aux fluctuations de la densit\'{e} du plasma avec une r\'{e}solution spatiale \'{e}lev\'{e}e. En particulier, la r\'{e}flectom\'{e}trie \`{a} fr\'{e}quence fixe a \'{e}t\'{e} largement utilis\'{e}e pour extraire les caract\'{e}ristiques spectrales et les propri\'{e}t\'{e}s de corr\'{e}lation des fluctuations turbulentes pour ensuite \'{e}tablir un lien entre les micro-instabilit\'{e}s et le transport turbulent.


En science de la fusion, la plupart des \'{e}tudes exp\'{e}rimentales sont usuellement organis\'{e}es en fonction d'un ensemble restreint de conditions plasma soigneusement choisies en tentant de faire varier de mani\`{e}re contr\^{o}l\'{e}e un ou plusieurs param\`{e}tres plasma \`{a} la fois pour \'{e}tudier l'effet de certains ph\'{e}nom\`{e}nes plasma consid\'{e}r\'{e}s comme pertinents. Dans ce travail de th\`{e}se une autre approche a \'{e}t\'{e} suivie en construisant une vaste base de donn\'{e}es \`{a} partir de mesures de r\'{e}flectom\'{e}trie effectu\'{e}es sur le tokamak Tore Supra. Pour cela des outils sp\'{e}cialis\'{e}s ont \'{e}t\'{e} d\'{e}velopp\'{e}s pour extraire les tendances globales persistantes dans l'ensemble des conditions plasma couvert par la base de donn\'{e}es de Tore Supra. Cette base contient 350 000 spectres provenant de 6000 d\'{e}charges qui sont, soit en r\'{e}gime "ohmique" o\`{u} le chauffage est uniquement assur\'{e} par le courant plasma, soit mode \`{a} faible confinement (mode L) o\`{u} un ou des syst\`{e}mes de chauffage auxiliaires sont actifs. L'objectif initial \'{e}tait d'extraire les tendances globales des propri\'{e}t\'{e}s de la turbulence via une analyse syst\'{e}matique des donn\'{e}es et de les relier \`{a} l'apparition et au comportement des micro-instabilit\'{e}s conduisant \`{a} la turbulence, contribuant ainsi \`{a} la compr\'{e}hension de la turbulence plasma.

Pour atteindre ces objectifs, un \'{e}l\'{e}ment cl\'{e} est l'\'{e}tablissement une param\'{e}trisation appropri\'{e}e du spectre en fr\'{e}quence de r\'{e}flectom\'{e}trie de fluctuation. Un sch\'{e}ma robuste de quantification du spectre passe par une normalisation qui permet aussi une analyse syst\'{e}matique initialement pr\'{e}vue dans ce travail. Tous les spectres de notre base de donn\'{e}es Tore Supra a \'{e}t\'{e} d\'{e}compos\'{e} en quatre composantes: la composante de courant continu (CC) correspondant \`{a} l'onde r\'{e}fl\'{e}chie, les fluctuations de basse fr\'{e}quence (BF), la turbulence \`{a} large bande (BB) et le niveau de bruit. Diff\'{e}rentes fonctions g\'{e}n\'{e}riques servant \`{a} la param\'{e}trisation ont \'{e}t\'{e} test\'{e}es et compar\'{e}es pour d\'{e}terminer l'ajustement optimal du spectre, avec des propri\'{e}t\'{e}s de robustesse ad\'{e}quates en pr\'{e}sence d'une grande vari\'{e}t\'{e} de formes de spectre et de conditions plasma. Les composantes des spectres aux basses fr\'{e}quences, y compris les composantes DC et BF, ont \'{e}t\'{e} ajust\'{e}es par deux fonctions gaussiennes et le bruit par un param\`{e}tre constant. Le composant BB repr\'{e}sente la distribution d'\'{e}nergie de la turbulence dans le domaine fr\'{e}quentiel et peut \^{e}tre identifi\'{e} diverses fonctions g\'{e}n\'{e}riques (gaussienne, lorentzienne, etc.) dans diff\'{e}rentes conditions de plasma. Pour la composante BB, trois mod\`{e}les d��ajustement diff\'{e}rents (gaussien g\'{e}n\'{e}ralis\'{e}, Voigt et Taylor) ont \'{e}t\'{e} compar\'{e}s quantitativement en utilisant un grand nombre de spectres. Cela indique une excellente performance du mod\`{e}le gaussien g\'{e}n\'{e}ralis\'{e} (GG), suivi du mod\`{e}le de Taylor. Par ailleurs, les param\`{e}tres du mod\`{e}le de Taylor sont plus facilement interpr\'{e}tables en termes de diff\'{e}rents effets physiques.

Dot\'{e}s d��une m\'{e}thode de param\'{e}trisation spectrale robuste, les param\`{e}tres spectraux (largeur, forme et contribution) des composants BB et LF des mod\`{e}les GG et Taylor fournissent des informations quantitatives sur les propri\'{e}t\'{e}s de turbulence. La caract\'{e}ristique spectrale la plus simple est la contribution des spectres \`{a} BB ($E_\mathrm{BB}$). Le profil radial complet de la composante BB a \'{e}t\'{e} \'{e}tudi\'{e} pour diff\'{e}rents facteurs de s\'{e}curit\'{e} ($q_{\psi}$) \'{e}valu\'{e}s au voisinage de la derni\`{e}re surface de flux magn\'{e}tique ferm\'{e}e. Ce param\`{e}tre sans dimension est d\'{e}terminant pour la stabilit\'{e} des plasmas contenus dans la base de donn\'{e}es. Dans les plasmas ohmiques, une r\'{e}duction remarquable de $E_\mathrm{BB}$, appel\'{e}e bassin de $E_\mathrm{BB}$, a \'{e}t\'{e} syst\'{e}matiquement observ\'{e}e pr\`{e}s de la r\'{e}gion centrale. Une analyse  plus approfondie a r\'{e}v\'{e}l\'{e} un lien direct entre le bassin $E_\mathrm{BB}$ et la surface $q = 1$. Ceci est li\'{e} \`{a} l'apparition d'instabilit\'{e}s magn\'{e}tohydrodynamiques connues sous le nom de dents de scie. Plus pr\'{e}cis\'{e}ment, $E_\mathrm{BB}$ est consid\'{e}rablement r\'{e}duit ($E_\mathrm{BB} <0,2 $) \`{a} l'int\'{e}rieur de la surface $q = 1$ et la largeur du bassin $E_\mathrm{BB}$ suit approximativement  les positions de  $q = 1$. En dehors de la surface $q = 1$, $E_\mathrm{BB}$ augmente au-dessus de $ 0,5 $ tant du c\^{o}t\'{e} champ bas que du c\^{o}t\'{e} champ haut, mais avec une forte asym\'{e}trie. En recherchant les diff\'{e}rences entre le r\'{e}gime de confinement ohmique lin\'{e}aire (LOC) et le r\'{e}gime de confinement ohmique satur\'{e} (SOC), il a \'{e}t\'{e} d\'{e}couvert que $E_\mathrm{BB}$ dans le r\'{e}gime SOC est syst\'{e}matiquement plus \'{e}lev\'{e} que dans le r\'{e}gime LOC, sur toute la travers\'{e}e du plasma. En mode L, nous nous sommes concentr\'{e}s sur les plasmas avec chauffage hybride inf\'{e}rieur pur (LH) ou chauffage par r\'{e}sonance cyclotron ionique pur (ICRH). Dans les plasmas chauff\'{e}s par la LH, le bassin $E_\mathrm{BB}$ a \'{e}galement \'{e}t\'{e} observ\'{e} pour diff\'{e}rentes puissances de chauffage ($P_\mathrm{heat}$). Cependant, avec le m\^{e}me $P_\mathrm{chaleur}$ dans les plasmas chauff\'{e}s seulement par ICRH, $E_\mathrm{BB}$ s'est r\'{e}v\'{e}l\'{e} \^{e}tre grandement am\'{e}lior\'{e}, avec un bassin \`{a} large qui tend \`{a} dispara\^{\i}tre.

Dans la derni\`{e}re partie du travail, l'interpr\'{e}tation des tendances spectrales observ\'{e}es dans la base de donn\'{e}es a \'{e}t\'{e} abord\'{e}e. Nous nous sommes concentr\'{e}s sur les effets des collisions sur diff\'{e}rentes caract\'{e}ristiques spectrales dans les diff\'{e}rents r\'{e}gimes de confinement. Dans les plasmas ohmiques, une augmentation g\'{e}n\'{e}rale de $E_\mathrm{BB}$ avec collision a \'{e}t\'{e} observ\'{e}e \`{a} toutes les positions radiales. Cette tendance globale a \'{e}t\'{e} jug\'{e}e compatible avec les simulations gyrocin\'{e}tiques rapport\'{e}es dans la litt\'{e}rature. Plus pr\'{e}cis\'{e}ment, une composante BB plus large a \'{e}t\'{e} observ\'{e}e dans le r\'{e}gime SOC (densit\'{e} plus \'{e}lev\'{e}e ou collisionalit\'{e}) par rapport au r\'{e}gime LOC. Cette correspondance sugg\`{e}re une interpr\'{e}tation possible des tendances de la largeur de bande large en termes de transition de l'instabilit\'{e} dominante \`{a} l'origine de la turbulence. En particulier, les instabilit\'{e}s TEM et ITG ont \'{e}t\'{e} li\'{e}es aux r\'{e}gimes LOC et SOC, respectivement. Cette interpr\'{e}tation possible a \'{e}t\'{e} bas\'{e}e par une analyse plus pouss\'{e}e de la composante de la FL et de la forme du profil de densit\'{e}. De plus, d'autres caract\'{e}ristiques de BB (largeur et forme) ont \'{e}galement \'{e}t\'{e} \'{e}tudi\'{e}es pour confirmer cette analyse. Dans les plasmas en mode L, une tendance similaire de $E_\mathrm{BB}$ avec une collisionalit\'{e} a \'{e}t\'{e} observ\'{e}e, sugg\'{e}rant une interpr\'{e}tation similaire \`{a} celle des cas Ohmique. L\`{a} encore, cela a \'{e}t\'{e} confirm\'{e} par une analyse globale de la composante de la FL et de la forme du profil de densit\'{e}.

La principale contribution du pr\'{e}sent travail a \'{e}t\'{e} d'\'{e}tendre l'analyse des caract\'{e}ristiques de la turbulence des spectres en fr\'{e}quence de r\'{e}flectom\'{e}trie \`{a} un large \'{e}ventail de conditions plasma pour une base de donn\'{e}es volumineuse. La standardisation offerte par cette approche permet une \'{e}tude syst\'{e}matique des propri\'{e}t\'{e}s de turbulence dans une base de donn\'{e}es qui pourrait \^{e}tre compl\'{e}mentaire \`{a} une analyse bas\'{e}e sur diff\'{e}rents mod\`{e}les restreints \`{a} des conditions plasma sp\'{e}cifiques correspondant \`{a} des observations exp\'{e}rimentales cibl\'{e}es ou \`{a} des simulations. Il s'agit de la premi\`{e}re d\'{e}monstration en science de la fusion de la caract\'{e}risation syst\'{e}matique des propri\'{e}t\'{e}s de turbulence \`{a} partir d'une base de donn\'{e}es regroupant plus de 10 ann\'{e}es d'exp\'{e}riences. Une confirmation suppl\'{e}mentaire du lien entre les tendances spectrales et l'instabilit\'{e} dominante devrait \^{e}tre apport\'{e}e par des simulations couplant code de propagation d'ondes et code gyrocin\'{e}tique. Malgr\'{e} tout le travail qui reste \`{a} effectuer, ce travail de th\`{e}se a permis d��\'{e}tablir des tendances robustes de premi\`{e}re importance concernant les propri\'{e}t\'{e}s de turbulence dans les plasmas de tokamak, confirmant l'interpr\'{e}tation existante \`{a} partir d'une analyse syst\'{e}matique originale.

}

%\newpage
%
%\section*{Samenvatting}
%
%Het onderzoek naar kernfusie door magnetische opsluiting is gericht op de ontwikkeling van een schone, veilige en onuitputtelijke basisvoorziening van elektriciteit. Gebaseerd op de energie ontwikkeld bij de samensmelting van lichte kernen van waterstofisotopen in een heet tokamakplasma, hangt de realisatie ervan cruciaal af van de mogelijkheid om het transport van energie en deeltjes doorheen het plasma te reguleren. Het is nu geweten dat de belangrijkste bijdrage tot het transport in een tokamakplasma te wijten is aan kleinschalige turbulentie. Het begrip en de controle van het turbulent transport speelt dan ook een sleutelrol in het bereiken van de vereiste opsluitingseigenschappen. In de centrale regio van tokamakplasma's is turbulentie door driftgolven het gevolg van twee types van microinstabiliteiten: de zogenaamde gevangen-elektronmodes (TEM) en de ion-temperatuurgradi\"entmodes (ITG). Aanzienlijke onderzoeksinspanningen zijn gericht op het begrip van de eigenschappen van turbulentie onder variabele plasmacondities. Het is bekend dat de botsingsgraad van het plasma de dominante instabiliteit be\"invloedt, omdat deeltjes die in hun beweging belemmerd worden door de magnetische spiegel, kunnen ontsnappen door botsingen. Deze mechanismen verduidelijken is \'e\'en van de hoofddoelstellingen van deze thesis.
%
%Onder de verschillende diagnostische instrumenten voor turbulentie, is reflectometrie een radar-achtige techniek, gebruikt in dit doctoraatswerk, gebaseerd op de detectie van de eigenschappen van microgolven die gereflecteerd worden door het plasma. Reflectometrie laat toe fluctuaties van de plasmadichtheid te detecteren met een hoge spatiale resolutie. In het bijzonder is de standaard reflectometrie bij vaste frequentie uitgebreid gebruikt om de spectrale karakteristieken en de correlatie-eigenschappen van turbulente fluctuaties te onttrekken, alsook de link met de micro-instabiliteiten en het turbulent transport.
%
%In de fusiewetenschappen worden de meeste experimentele studies opgezet rond een zorgvuldig gekozen set plasmacondities, waarbij \'e\'en of meerdere plasmaparameters op een gecontroleerde manier gevarieerd worden, om zo het effect te onderzoeken op een plasmafenomeen dat het onderwerp is van de studie. In dit werk daarentegen, werd een grote databank van reflectometriemetingen van de Tora Supra-tokamak gecre\"eerd, en werden gespecialiseerde technieken ontwikkeld om patronen te detecteren in de data, die zich voordoen over een groot bereik van plasmacondities. De databank bevat 350.000 spectra van 6.000 ontladingen in plasma's in de basis-Ohmse operatie, en in de zogenaamde lage-opsluitingsmode (L-mode), waarbij verhitting toegepast wordt met bijkomende verhittingssystemen. Het doel was algemene trends te vinden in de eigenschappen van de turbulentie, en om deze te linken met het voorkomen en het gedrag van micro-instabiteiten die de turbulentie aansturen, om op deze manier bij te dragen aan het begrip van plasmaturbulentie.
%
%Om deze doelstellingen te bereiken, is het cruciaal de vermogensspectra van de reflectometrie bij vast frequentie op een gedegen manier te kwantificeren. Een robuuste methode om de spectra te kwantificeren maakt standaardisering mogelijk, waardoor op zijn beurt de soort systematisering mogelijk wordt die beoogd wordt in dit werk. Elk spectrum in onze Tore Supra-databank werd ontbonden in vier componenten: de gelijkstroomcompnent (DC), de component van laag-frequente fluctuaties (LF), de breedbandcomponent (BB) en het ruisniveau. Verschillende parametriserende functies werden getest em vergeleken, om de optimale fit van de spectra te bepalen, met adequate robuustheidseigenschappen in de aanwezigheid van sterk vari\"erende spectrale vormen en plasmacondities. De componenten van de spectra bij lage frequenties, inclusief de DC- en LF-componenten, werden gefit door twee Gaussiaanse functies, en de ruis werd gefit door een constante parameter. De BB-component stelt de energieverdeling voor van de turbulentie in het frequentiedomein, en kan verschillende vormen aannemen (Gaussiaans, Lorentziaans, enz.) onder verschillende plasmacondities. Drie verschillende fitmodellen (veralgemeende Gaussiaan, Voigt en Taylor) werden kwantitatief vergeleken voor de BB-component, gebruik makend van een groot aantal spectra. Dit wees op een excellente performantie van de veralgemeende Gaussiaan (GG), gevolgd door het Taylormodel. Anderzijds lenen de parameters van het Taylormodel zich beter tot fysische interpretatie.
%
%Uitgerust met een robuuste methode voor de parametrisering van de spectra, leveren de spectrale parameters (breedte, vorm en bijdrage) van de BB- en LF-componenten van zowel het GG- als het Taylormodel kwantitatieve informatie over de eigenschappen van de turbulentie. De meest voor de hand liggende karakteristiek is de BB-bijdrage tot het spectrum ($E_\mathrm{BB}$). Het volledige radiale profiel van de BB-component werd onderzocht bij verschillende waarden van de rand-veiligheidsfactor ($q_{\psi}$), een cruciale dimensieloze parameter die de plasmastabiliteit bepaalt, zowel in plasma's in de Ohmse-mode als de L-mode. In Ohmse plasma's werd systematisch een opmerkelijke reductie van $E_\mathrm{BB}$ vastgesteld, het $E_\mathrm{BB}$-basin genoemd, in het centrale gebied van het plasma. Nader onderzoek onthulde een directe link tussen het $E_\mathrm{BB}$-basin en het $q=1$-oppervlak. Dit heeft te maken met het voorkomen van magnetohydrodynamische instabiliteiten bekend als zaagtanden. In het bijzonder was $E_\mathrm{BB}$ sterk gereduceerd ($E_\mathrm{BB} < 0.2$) binnen het $q=1$-oppervlak en de breedte van het $E_\mathrm{BB}$-basin werd bij benadering evenredig bevonden met de de $q=1$-positie. Buiten het $q=1$-oppervlak steeg $E_\mathrm{BB}$ boven $0.5$, zowel aan de lage-veldzijde als de hoge-veldzijde, hoewel met een sterke asymetrie. Wanneer we onderscheid maken tussen het lineaire Ohmse opsluitingsregime (LOC) en het gesatureerde Ohmse opsluitingsregime (SOC), werd ontdekt dat $E_\mathrm{BB}$ in het SOC-regime systematich hoger lag dan in het LOC-regime, doorheen de volledige plasmadoorsnede. In L-mode focusten we op plasma's met pure lage-hybride-verhitting (LG) of pure ion-cyclotron-resonantieverhitting (ICRH). In pure LH-verhitte plasma's werd het $E_\mathrm{BB}$-basin eveneens geobserveerd, bij verschillende verhittingsvermogens ($P_\mathrm{heat}$). Bij gelijke $P_\mathrm{heat}$ in pure ICRH-plasma's werd echter een aanzienlijk hogere $E_\mathrm{BB}$ vastgesteld, met een zwak of afwezig breedbandbasin.
%
%In het finale deel van het werk werd de interpretatie van de observaties van de spectrale trends doorheen de databank behandeld. We concentreerden ons op botsingseffecten op de verschillende spectrale karakteristieken in de verschillende opsluitingsregimes. In Ohmse plasma's werd een algemene toename van $E_\mathrm{BB}$ vastgesteld met botsingsgraad op alle radiale posities. Deze globale trend werd consistent bevonden met gyrokinetische simulaties vermeld in de literatuur. In het bijzonder werd een bredere BB-component waargenomen in het SOC-regime (hogere dichtheid of botsingsgraad), vergeleken met het LOC-regime. Deze overeenstemming suggereert een mogelijk interpretatie van de trends van de breedbandbreedte in termen van een transitie van de dominante instabiliteit die de turbulentie drijft. In het bijzonder zijn de TEM- en ITG-instabiliteiten in het verleden gelinkt aan het LOC- en SOC-regime, respectievelijk. Deze mogelijke intrepretatie werd ondersteund door bijkomende analyse van de LF-component en de dichtheidspiekfactor. Daarnaast werden andere BB-karakteristieken (breedte en vorm) eveneens onderzocht, om tot een dieper begrip te komen. In L-mode-plasma's werd een gelijkaardige trend waargenomenen van $E_\mathrm{BB}$ met de botsingsgraad, wat wijst op een gelijkaardige interpretatie als in het Ohmse geval. Opnieuw werd dit ondersteund door een globale analyse van de LF-component en de dichtheidspiekfactor.
%
%De belangrijkste bijdrage van dit werk is de uitbreiding geweest van de analyse van turbulentiekarakteristieken van reflectometriespectra naar een groot bereik van plasmacondities in een grote databank. De standaardisering die deze aanpak biedt, maakt een systematische studie mogelijk van turbulentie-eigenschappen doorheen de databank. Verschillende patronen werden waargenomen en konden gelinkt worden aan experimentele obsevaties of simulaties uitgevoerd met een sterk gereduceerde set van plasmacondities. Dit is de eerste demonstratie in de fusiewetenschappen van een systematische karakterisering van turbulentie-eigenschappen in een dergelijke grote databnk. Bijkomende bevestiging van de link tussen de spectrale trends en de dominerende instabiliteit zal geleverd moeten worden door volledige-golf-simulaties en gyrokinetische simulaties. Niettemin heeft de huidige studie toegelaten een aantal belangrijke, robuuste trends vast te leggen van turbulentie-eigenschappen in tokamakplasma's, om op die manier hun interpretatie verder te onderbouwen.


