
\chapter*{Summary}
\addcontentsline{toc}{chapter}{Summary}


Turbulence is an important issue in fusion plasmas physics research as it was found to have a direct link to the particles and energy transports, and hence the confinement performance. In the core region of tokamaks, the drift-wave turbulence includes two main micro-instabilities: the trapped electron modes (TEM) and the ion temperature gradient (ITG) modes. It is important to understand the turbulence properties in different plasma conditions and distinguish the various instabilities in these conditions. In particular, the collisionality is regarded as a crucial parameter to affect the dominating instabilities, as the trapped particles tend to be detrapped by collisions and thus the TEM instability could be reduced with increasing collisionality. Among the different turbulence measurement tools, reflectometry is an active microwave diagnostic detecting the phase (the density) fluctuations at low wavenumber with high spatial resolution. In particular, the standard fixed-frequency reflectometry has been extensively utilized to extract spectral characteristics and correlation properties of turbulent fluctuations and their connections with the micro-instabilities and the turbulent transport. Different from the conventional shot-to-shot analysis validating by some dedicated simulations, this thesis aims at extracting general trends of turbulence properties and the link to the micro-instabilities through a systematic (database) analysis of a large refletometry database. The database includes 350,000 spectra from 6,000 discharges in the Ohmic and low-confinement mode (L-mode) plasmas.


Based on the previous decomposition work of the turbulence (power and coherence) spectra, a parametrization method of frequency spectra from fixed-frequency reflectometry data on the Tore Supra tokamak has been developed. For a systematic analysis with the most fundamental turbulence properties, each spectrum was decomposed into four components: the direct current (DC) component, the low-frequency (LF) fluctuations, the broadband (BB) turbulence and the noise level. Different kinds of parameterized functions have been tested and compared to determine the optimal parametrization (spectrum fit) model. The low frequency parts of the spectra, including the DC and LF components, were fitted by two Gaussian functions, and the noise was fitted by one constant parameter. The BB component represents the energy distribution of turbulence in frequency domain and could change between variety of shapes (Gaussian, Lorentzian, etc.) with varying plasma conditions. For the BB component, three different fit models (generalized Gaussian, Voigt and Taylor) were compared quantitatively using the residual sum of squares (RSS) and the Bayesian information criterion (BIC) by a large number of spectra. Both RSS and BIC indicate statistically the excellent performance of the generalized Gaussian (GG) and the Taylor model, with the GG model even more superior. Furthermore, the qualitative comparisons based on some representative spectral shapes confirms the outstanding fit performance of the GG and Taylor model, with the Taylor model has a better decomposition from a physical viewpoint. 


From the parametrization, the spectral parameters (width, shape and contribution) of the BB and LF components from both GG and Taylor model provide quantitative characteristics of turbulence properties. The most straightforward spectral characteristic is the BB contribution of the spectra ($E_\mathrm{BB}$), whose complete radial profiles have been investigated at different edge safety factor ($q_{\psi}$), a crucial dimensionless parameter determining the plasma stability, in both Ohmic and L-mode plasmas. In Ohmic plasmas, a remarkable reduction of $E_\mathrm{BB}$, referred as the $E_\mathrm{BB}$ basin, was systematically observed near the central region. Further investigations has revealed a direct link between the $E_\mathrm{BB}$ basin and the $q = 1$ surface, an important radial position relating to some magnetohydrodynamic instabilities such as the sawteeth activities. Specifically, $E_\mathrm{BB}$ was tremendously reduced ($E_\mathrm{BB} < 0.2$) inside the $q = 1$ surface and the width of the $E_\mathrm{BB}$ basin is approximately proportional to the $q = 1$ position. Outside the $q = 1$ surface, $E_\mathrm{BB}$ increases above $0.5$ at both the low filed side (LFS) and the high field side (HFS), with a strong asymmetry exists between the LFS and the HFS. Furthermore, a deeper investigation has distinguished the Ohmic plasmas into two confinement regimes: the linear Ohmic confinement (LOC) regime and the saturated Ohmic confinement (SOC) regime, by an empirical scaling law extracted from the Tore Supra database. It was discovered that $E_\mathrm{BB}$ in the SOC regime is globally higher than in the LOC regime at different radial positions and $q_{\psi}$. In L-mode plasmas, the systematic study has been focused on the plasmas with pure lower hybrid (LH) heating or pure ion cyclotron resonance heating (ICRH) method. In pure LH heating plasmas, the $E_\mathrm{BB}$ basin was also observed inside the $q = 1$ surface at different heating power ($P_\mathrm{heat}$). However, with the same $P_\mathrm{heat}$ in pure ICRH plasmas, $E_\mathrm{BB}$ is greatly increased and thus the $E_\mathrm{BB}$ basin becomes much weaker or even disappears. 


To further understand the behaviors of $E_\mathrm{BB}$ in different confinement regimes or heating methods, the collisional effects on different spectral characteristics have been studied for both Ohmic and L-mode plasmas. In Ohmic plasmas, a general increase of $E_\mathrm{BB}$ with respect to the collisionality was observed globally at different radial positions. This global trend is consistent with the previous gyrokinetc simulations where a wider BB component was observed in the SOC regime (higher density or colisionality) than in the LOC regime. This consistency further drives us to propose a possible interpretation to explain the modifications of $E_\mathrm{BB}$ by a change of the instabilities, i.e., the TEM and the ITG instabilities, which have been accepted to be linked with the LOC and the SOC regimes, respectively. This possible interpretation was supported by further analysis of the LF component and the density peaking. In addition, other BB characteristics (width and shape) has have also been studied to have a deeper understanding. In L-mode plasmas, the similar evolutions of $E_\mathrm{BB}$ with the collisionality was observed, which indicates a possible similar interpretation as the Ohmic cases. This was again backed by the global analysis of the LF component and the density peaking. However, other parameters show different behaviors compared with the Ohmic plasmas and their general trends could be difficult to be extracted, which indicates a complicated and still unexplained underlying physical nature. In order to further confirm the link between the modifications of the spectra and the change of the instabilities for both Ohmic and L-mode plasmas, more more sophisticated studies by full-wave and gyrokinetic simulations are required.
